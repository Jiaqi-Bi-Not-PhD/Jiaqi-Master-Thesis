\chapter{Methods}
\section{Introduction}
When making the imputation on the continuous variable, one common way is to assume a conditionally normal distribution. 
This conditional distribution can be estimated from a linear regression. 
The original multiple imputation structure was brought by~\citet{rubin1987multiple}, that this method has been widely used by different scientific researchers with different models.
There are many softwares that based on the multiple imputations as well, such as Blimp software~\cite{keller2021blimp}, \texttt{MICE} package in R~\cite{royston2011multiple}, and \texttt{jomo} package in R~\cite{quartagno2019jomo}. 
Apparently, this thesis cannot provide enough rooms for those well-designed softwares that I have not mentioned. 
The development of Multiple Imputation within certain specific models remains incomplete, and many details have not yet been comprehensively addressed in the current published literature.
In genetic epidemiology, studies are mostly conducted with family clusters. 
Current implementation of the multiple imputation does not account for the genetic correlations. 
Furthermore, there remains room for exploration in the application of the frailty model with ascertainment correction.
Therefore, this research is designed to develop a computationally efficient multiple imputation method to account for the kinship correlations, and apply it to frailty models with ascertainment correction. 

In this chapter, a comprehensive guideline and adjusted multiple imputation formulas are provided. 
We explicitly show how the kinship matrix works in the imputation step, and how the ascertainment correction handles the analysis step in this research. 
The special imputation model is introduced as well, while considering the genetic variance and residual variance (sometimes refer to the environmental variance).
The variance estimation using Rubin's Rule is provided, as well as the confidence interval based on completed data following the proposed multiple imputation.

\section{Kinship Matrix}

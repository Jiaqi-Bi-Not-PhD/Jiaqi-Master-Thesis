\chapter{Literature Reviews}
\section{Survival Analysis}
Survival analysis is a robust statistical methodology used to analyze time-to-event data, where the focus is on the time until an event of interest occurs.
It has been extensively applied in medical research, particularly in studies involving cancer, where events such as death or relapse are critical endpoints.
The literature identifies several key methods and models that form the backbone of survival analysis. \citet{kaplan1958nonparametric} introduced the Kaplan-Meier estimator, a nonparametric statistic used to estimate survival functions from incomplete observations, which remains widely used due to its simplicity and effectiveness in handling censored data.
\citet{cox1972regression} proposed the proportional hazards model, which allows for the inclusion of covariates and has become a standard technique for assessing the effect of explanatory variables on survival.
\citet{collett2023modelling}, \citet{machin2006survival}, and \citet{kleinbaum1996survival} provide detailed expositions on survival analysis methods, including parametric and nonparametric approaches.
Recent advancements have addressed complex issues such as interval censoring and competing risks, expanding the applicability and precision of survival analysis.
hese methodological developments have significantly enhanced the ability to make informed inferences from survival data, contributing to more accurate prognostic assessments and treatment evaluations in clinical research.
In the survival analysis, there are several key functions that will contribute essentially to nearly all relevant scientific work. 

The survival function $S(t)$ measures the probability of study subject's entry age $t$ is less than the event time $T$, 
\begin{equation} 
    S(t)=P(T>t)
\end{equation}
The cumulative distribution function $F(t)$ represents the probability that the event has occurred by time $t$. 
This is the complement of the survival function $S(t)$,
\begin{equation}
    F(t)=P(T\leq t)=1-S(t)
\end{equation}
The probability density function $f(t)$ is the likelihood of the event occurring at an exact time $t$. 
It is the derivative of the cumulative distribution function (CDF) or the negative derivative of the survival function. 
\begin{equation}
    f(t)=\frac{d}{dt}F(t)=-\frac{d}{dt}S(t)=h(t)S(t)
\end{equation}
The hazard function $h(t)$ measures the instantaneous rate at which the event occurs, given that the individual has survived up to the time $t$. 
It is the probability that an event occurs in a very small time interval, given survival until the beginning of the interval. 
\begin{equation}
    h(t)=\lim_{\Delta t\rightarrow 0}\frac{P(t\leq T<t+\Delta t|T\geq t)}{\Delta t}=\frac{f(t)}{S(t)}
\end{equation}
The accumulated risk of experiencing the event up to time $t$ can be expressed using the cumulative hazard function. 
It is the integral of the hazard function over time, providing a cumulative measure of risk. 
\begin{equation}
    H(t)=\int_0^t h(u)du=-\log S(t)
\end{equation}
Whenever we can define one of above functions, it is straighforward to derive the rest. 




\subsection{Frailty Model for Family Based Study}
Parametric survival analysis methods assume that the time-to-event data follow a specific probability distribution.
This approach provides a more detailed and flexible framework for modeling survival data, allowing for more precise estimates and interpretations of survival functions and hazard rates.
\citet{hosmer2008applied} has comprehensively summarized most of parametric baseline hazard
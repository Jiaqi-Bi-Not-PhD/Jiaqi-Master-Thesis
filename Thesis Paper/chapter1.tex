\chapter{Introduction}
\section{Background}
The Breast Cancer type 1/2, usually referred as BRCA1/2, are proteins that consists of genes that code for BRCA1 in humans. BRCA1/2 are human tumor suppressor genes, that are responsible for repairing the DNA~\cite{duncan1998brca1}. 
When the mutation exists on these genes may cause the impairments of proper functions, which can lead to the possibility of capturing the breast, ovarian, or other specific cancers~\cite{haffty2002outcome, huang2018association}. 
Inheriting one of these mutations does not guarantee developing cancer disease, but the mutation can increase the risk of getting those cancers. 

In the field of medicine, these cancer types are classified as Hereditary Breast and Ovarian Cancer Syndrome (HBOC). 
The average life expectancy of individuals with BRCA1, without any interventions, is approximately 4.2 years shorter than that of non-carriers of the BRCA1 gene~\cite{lux2006hereditary}.
Significant advancements have been made in the medical and statistical modeling of breast cancer risk among BRCA1/2 carriers. These include the application of competing risk survival analysis based on breast and ovarian cancer outcomes developed by~\citet{choi2021competing}, as well as various clinical trials investigating risk-reducing treatment approaches for breast cancer patients~\cite{choi2021association}.
Despite these efforts, it remains crucial to ensure statistical validity across these studies especially when missing data exists.

From a statistical perspective, the study is centered on a specific disease, which may introduce selection bias due to the sampling process. 
This bias arises from the selection criteria based on specific probands in each family. 
To mitigate this sampling bias, an ascertainment correction should be applied to the likelihood calculation, conditioning on the proband information.
To accurately capture the heterogeneity between families in the context of time-to-cancer outcomes, the use of a frailty model is recommended. 
There are various choices for frailty distributions in survival analysis, including the Gamma distribution and the log-Normal distribution.

\section{Motivation}
Although numerous studies on risk assessment in susceptible populations and statistical advancements in dynamic prediction have significantly contributed to understanding BRCA1/2 families, the issue of missing data remains a substantial challenge, particularly in the context of survival outcomes.
Over the past decade, several methodologies have been proposed to address missing data, including the Expectation-Maximization (EM) algorithm, the Monte-Carlo EM algorithm for cases where the E-step lacks a closed form, and Multiple Imputation (MI). 
However, when applying frailty models, which incorporate random effects in survival analysis, the literature addressing missing data is relatively sparse.

In genetic epidemiology, research is typically conducted on a family-wise basis.
Therefore, considering the family structure when addressing statistical problems is both essential and unavoidable. 
Moreover, existing techniques for handling missing data must be carefully adapted, as the clustered nature of the dataset introduces additional complexity.
Within the genetic framework, many variables, such as genetic information and polygenic risk scores (PRS), are not independent between individuals. 
Traditional methodologies often fail to account for family correlations and ascertainment bias. 
Given that families are selected based on a proband, it is crucial to apply ascertainment correction to minimize the selection bias.
This situation presents an opportunity to further investigate and develop adequate methods for handling missing data, taking into account family correlations and ascertainment bias.

In this project, we aim to investigate the current implementation of Multiple Imputation (MI) methods for frailty models. 
Additionally, we propose a novel MI method that explicitly incorporates the kinship matrix during the imputation of genetically related variables. 
This proposed method will be evaluated by comparing it to existing MI methods and Complete Case Analysis (CCA).

\section{Objectives}
With the proposed MI method and the BRCA1 data, the objectives of this thesis are designed as follows: 
\begin{enumerate}
    \item To adapt the kinship correlations into the imputation step
    \item To incorporate the ascertainment correction into the likelihood while considering that not all probands are affected
    \item To assess the novel MI method via the calculation of the estimations, biases, and precisions through the simulation study
    \item To apply the novel MI method and adjusted likelihood to model the BRCA1 family data
\end{enumerate}

\section{Organizations of the Thesis}

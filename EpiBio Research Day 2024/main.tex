\documentclass [aspectratio=169]{beamer}
\beamertemplatenavigationsymbolsempty
\usetheme{Boadilla}
\usepackage{textpos} % package for the positioning
\usepackage[]{graphicx}
\usepackage{graphicx}
\usepackage{float}
\usepackage{hyperref}
\usepackage{caption}
\usepackage{subcaption}
\usepackage{algorithm,algpseudocode}
\usepackage[export]{adjustbox}
\usepackage{tikz}
%\usepackage[square,numbers]{natbib}
\usepackage[square,numbers]{natbib}
\usepackage[byname]{smartref}
\usetikzlibrary{positioning}
\usetikzlibrary{arrows, shapes, decorations, automata, backgrounds, fit, petri, calc}

\newcommand*{\logofont}{\fontfamily{phv}\selectfont}

\definecolor{uwopurple}{RGB}{79,38,131} % official blue color for uoft

\title[]{\vspace{60pt} \\
Correlated Shared Frailty Model
Incorporating Ascertainment Correction with Missing Covariates
in Family-Based Studies}
%\subtitle{n}
\author[]{Jiaqi Bi, Osvaldo Espin-Garcia, Yun-Hee Choi}
\institute[]{Department of Epidemiology and Biostatistics\\
             Schulich School of Medicine \& Dentistry\\
             University of Western Ontario}
\date{\today}

% Math notations
\newtheorem{thm}{Theorem}[section]
\newtheorem{lem}[thm]{Lemma}

\newtheorem{defn}[thm]{Definition}
\newtheorem{eg}[thm]{Example}
\newtheorem{ex}[thm]{Exercise}
\newtheorem{conj}[thm]{Conjecture}
\newtheorem{cor}[thm]{Corollary}
\newtheorem{claim}[thm]{Claim}
\newtheorem{rmk}[thm]{Remark}

\newcommand{\ie}{\emph{i.e.} }
\newcommand{\cf}{\emph{cf.} }
\newcommand{\into}{\hookrightarrow}
\newcommand{\dirac}{\slashed{\partial}}
\newcommand{\R}{\mathbb{R}}
\newcommand{\C}{\mathbb{C}}
\newcommand{\Z}{\mathbb{Z}}
\newcommand{\N}{\mathbb{N}}
\newcommand{\Q}{\mathbb{Q}}
\newcommand{\LieT}{\mathfrak{t}}
\newcommand{\T}{\mathbb{T}}
\newcommand{\A}{\mathds{A}}
\newcommand{\E}{\mathbb{E}}
\newcommand{\Prob}{\mathbb{P}}
\newcommand{\Var}{\text{Var}}
\newcommand\equalhat{%
\let\savearraystretch\arraystretch
\renewcommand\arraystretch{0.3}
\begin{array}{c}
\stretchto{
    \scalerel*[\widthof{=}]{\wedge}
    {\rule{1ex}{3ex}}%
}{0.5ex}\\ 
=%
\end{array}
\let\arraystretch\savearraystretch
}

% set color
\setbeamercolor{title in head/foot}{bg=white}
\setbeamercolor{author in head/foot}{bg=white}
\setbeamercolor{date in head/foot}{fg=uwopurple}
\setbeamercolor{date in head/foot}{bg=white}
\setbeamercolor{title}{fg=uwopurple}
\setbeamerfont{title}{series=\bfseries}
\setbeamercolor{frametitle}{fg=uwopurple}
\setbeamerfont{frametitle}{series=\bfseries}
\setbeamercolor{block title}{bg=uwopurple!30,fg=black}
\setbeamercolor{item}{fg=uwopurple}
\setbeamercolor{caption name}{fg=uwopurple!70!}





% set logo at non-title pages
\logo{\includegraphics[height=0.9cm]{schulichuwo.png}\vspace*{-.055\paperheight}\hspace*{.85\paperwidth}}

\begin{document}

{
\setbeamertemplate{logo}{}
\begin{frame}
    \titlepage
    \begin{textblock*}{4cm}(0.5cm,-7.3cm)
        \includegraphics[width=4cm]{schulichuwo.png}
    \end{textblock*}
    \begin{textblock*}{8cm}(5.0cm,-7.0cm)
        \huge \color{uwopurple}{$\Bigr\rvert$ \hspace{0.15cm} \textbf{SSC 2024}}
    \end{textblock*}
\end{frame}
}

\begin{frame}{Background}
    \begin{block}{Breast Cancer}
        \begin{itemize}
            \item There were estimated 30,500 new cases of breast cancer in Canada in 2024, and approximately 5,500 deaths, making it the second leading cause of cancer-related death among women~\cite{BCCanadaStatistics2023}.
            \item Hereditary breast-ovarian cancer (HBOC) is an autosomal dominant disease characterized by germline pathogenic mutations in the BRCA1/2 genes~\cite{pritchard2019new}.
            \item Time-To-Cancer as an outcome, mutation gene status (mgene) \& Polygenic Risk Score (PRS) are predictors - Problems: There are missing data!
        \end{itemize}
    \end{block}
\end{frame}

\begin{frame}{Background}
    \begin{block}{Frailty Model for Family Data}
        \begin{itemize}
            \item Many different frailty models have been proposed for the analysis of BRCA1/2 families by~\citet{choi2021competing, chen2009frailty}
            \item Missing data remains a problem
        \end{itemize}
    \end{block}
    \begin{block}{Missing Data}
        \begin{itemize}
            \item The issue of the missing data was firstly brought by~\citet{rubin1976inference} in 1976.
            \item Three missing mechanisms: MCAR, Missing At Random (MAR), Missing Not At Random (MNAR)
        \end{itemize}
    \end{block}
\end{frame}

\begin{frame}{Survival Analysis}
    \begin{block}{Survival Analysis}
        The hazard function is defined as 
        \begin{equation} 
            h_{ij}(t_{ij}|\mathbf{x}_{ij}, z_j)=h_0(t_{ij})\exp(\boldsymbol{\beta}\mathbf{x}_i)z_j
        \end{equation}
        In the parametric setting, the baseline hazard $h_0(t_{ij})$ has a closed form. 
    \end{block}
    \begin{block}{Frailty Term}
        The dataset is clustered, so a frailty is required when modelling the time-to-event outcome to introduce random effects, association and unobserved heterogeneity.
        \begin{equation} 
            z_j\sim \text{Gamma}(\kappa, \kappa);~ z_j\sim\text{log-Normal}(0, \kappa^2)
        \end{equation}
        where $\kappa$ is the shape and rate parameters for Gamma distribution. 
    \end{block}
\end{frame}

\begin{frame}{Conditional Likelihood}
    \begin{block}{When there is no missing data}
        For individual $i$ in family $j$,
        \begin{align} 
            L(\boldsymbol{\theta}|z_j)=\prod_{j=1}^J\prod_{i=1}^{n_j} h(t_{ij}|\mathbf{x}_{ij}, z_j)^{\delta_{ij}}\exp (-H(t_{ij}|\mathbf{x}_{ij},z_j))
        \end{align}
    \end{block}
    \begin{block}{Ascertainment Correction}
        Denote $A(\boldsymbol{\theta})$ be the likelihood for the proband, and $p_j$ be the proband in family $j$, also denote $I(T_{p_j}<a_{p_j})$ as an indicator of the proband was affected before their entry to the study 
        \begin{equation} 
            A(\boldsymbol{\theta})=\Big [1-S_{p_j}(a_{p_j}|\mathbf{x}_{p_j})\Big ]^{I(T_{p_j}<a_{p_j})}S_{p_j}\Big [(a_{p_j}|\mathbf{x}_{p_j})\Big ]^{1-I(T_{p_j}<a_{p_j})}
        \end{equation} 
        Because not all probands are affected.
    \end{block}
\end{frame}

\begin{frame}{Conditional Likelihood}
    \begin{block}{Ascertainment Correction}
        Then the complete conditional likelihood becomes
        \begin{align} 
            L_C(\boldsymbol{\theta})=\frac{L(\boldsymbol{\theta})}{A(\boldsymbol{\theta})}
        \end{align}
    \end{block}
\end{frame}

\begin{frame}{Current multiple imputation methods (Continuous variable)} 
    \begin{enumerate} 
        \item Calculate $\hat{y}=\hat{\boldsymbol{\beta}}\mathbf{x}$ using $y_{obs}$, and $\hat{\boldsymbol{\beta}}$ can be obtained easily, as well as $\hat{\sigma}$, and $\text{Var}(\hat{\boldsymbol{\beta}})=\mathbf{V}$
        \item Draw $g\sim \chi^2_{n_{obs}-p}$ for one random draw 
        \item Calculate $\sigma^*=\hat{\sigma}/\sqrt{SSE/g}$
        \item Draw a $p$ dimensional vector $\mathbf{u}_1$ such that $u_{1k}\stackrel{iid}{\sim} N(0,1)$ and $k=1,...,p$
        \item Calculate $\boldsymbol{\beta}^*=\hat{\boldsymbol{\beta}}+\frac{\sigma^*}{\hat{\sigma}}\mathbf{u}_1\mathbf{V}^{1/2}$ such that $\mathbf{V}^{1/2}$ is the cholesky decomposition of $\mathbf{V}$
        \item Draw $u_{2i}\stackrel{iid}{\sim} N(0,1)$ 
        \item Impute $y_{mis,i}=\boldsymbol{\beta}^*\mathbf{x}_i+u_{2i}\sigma^*$ 
        \item Repeat 2. to 7. for $M$ times to obtain $M$ complete datasets
    \end{enumerate}
\end{frame}

\begin{frame}{Challenges on current MI methods}
    It fails to account for the kinship and frailty. 
\end{frame}

\begin{frame}{Kinship Matrix}
    In the genetic epidemiology, covariates are often genetically correlated within one family, current MI for continuous data assumes
    \begin{equation} 
        x_{ij,mis}|\mathbf{x}_{ij,obs}\sim N(\boldsymbol{\beta}\mathbf{x}_{ij,obs}, \sigma^2) 
    \end{equation} 
    Is this an adequate assumption? 
\end{frame}

\begin{frame}{Kinship Matrix}
    An example of a kinship matrix in a family, suppose there are 4 individuals...
    \[
K = \begin{bmatrix}
1    & 0.5  & 0.25 & 0 \\
0.5  & 1    & 0.5  & 0 \\
0.25 & 0.5 & 1    & 0 \\
0    & 0    & 0    & 1 \\
\end{bmatrix}
\]
\begin{itemize} 
    \item In $K_{11}$,the individual is fully related to themselves. 
    \item In $K_{12}=0.5$, the first and second individuals are half-related (Siblings). 
    \item In $K_{13}=0.25$, the first and third individuals are a quarter-related (half-siblings or grandparent-grandchild). 
    \item In $K_{14}=0$, the first and the fourth individuals are not related. 
\end{itemize}
\end{frame}

\begin{frame}{Kinship Matrix}
    Accounting for the genetic effects (kinship),
    \begin{equation}
        \mathbf{x}_{j,mis,1}|\mathbf{x}_{j,obs}\sim MVN(\boldsymbol{\beta}\mathbf{x}_{j,obs}, \sigma_g^2K+\sigma_e^2I)
    \end{equation}
    such that $K$ is the kinship correlation matrix with diagonal of 1. 
    \begin{itemize} 
        \item $\sigma_g^2$ accounts for the genetic variances, and $\sigma_e^2$ accounts for the residual variances.
        \item The multivariate normal distribution is what we are sampling the missing PRS when considering the kinship matrix. 
        \item Denote $\boldsymbol{\Sigma}=\sigma_g^2K+\sigma_e^2I$.
    \end{itemize}
\end{frame}

\begin{frame}{Multivariate Normal}
    \begin{block}{Multivariate Normal}
        Assume $\boldsymbol{\mu}=\boldsymbol{\beta}\mathbf{x}_{ij,obs}$. For each individual $i$, partition the data into 
        \begin{equation*}
        \begin{pmatrix} 
            \mu_i \\
            \boldsymbol{\mu}_{-i}
        \end{pmatrix}
    \end{equation*}
    The covariance can also be partitioned to 
    \begin{equation*} 
        \boldsymbol{\Sigma}=
        \begin{pmatrix} 
            \Sigma_{ii} & \boldsymbol{\Sigma}_{i,-i} \\
            \boldsymbol{\Sigma}_{-i,i} & \boldsymbol{\Sigma}_{-i,-i}
        \end{pmatrix}
    \end{equation*}
    so the conditional expectation can then be derived to 
    \begin{equation} 
        E(x_{mis,i}|\mathbf{x}_{-i})=\mu_i^*+\hat{\boldsymbol{\Sigma}}_{i,-i}\hat{\boldsymbol{\Sigma}}_{-i,-i}^{-1}(\mathbf{y}_{-i}-\boldsymbol{\mu}^*_{-i})
    \end{equation}
    \end{block}
\end{frame}

\begin{frame}{MI with Kinship Matrix}
    \begin{enumerate}
        \item Obtain the kinship matrix $K$ among all individuals 
        \item Calculate the estimates of $\hat{y}=\mathbf{x}\hat{\boldsymbol{\beta}}$, obtain estimates of $\hat{\boldsymbol{\beta}}$, $\hat{\sigma_g}^2$, $\hat{\sigma_e}^2$, $\text{Var}(\hat{\boldsymbol{\beta}})=\mathbf{V}$. In this step, naturally, $\boldsymbol{\hat{\Sigma}}$ is obtained.
        \item Draw $p$-dimensional vector $w_1$ such that $w_{1k}\stackrel{iid}{\sim} N(0,1)$ where $k=1,...,p$
        \item Calculate $\boldsymbol{\beta}^*=\hat{\boldsymbol{\beta}}+w_1\mathbf{V}^{1/2}$ such that $\mathbf{V}^{1/2}$ is the cholesky decomposition of $\mathbf{V}$
        \item Obtain $\mu_i^*=\boldsymbol{\beta}^*\mathbf{x}_i$
        \item Obtain the conditional expectations 
        \begin{equation}
            E(y_{mis,i}|\mathbf{y}_{-i})=\mu_i^*+\hat{\boldsymbol{\Sigma}}_{i,-i}\hat{\boldsymbol{\Sigma}}_{-i,-i}^{-1}(\mathbf{y}_{-i}-\boldsymbol{\mu}^*_{-i})
        \end{equation}
        \item Impute $y_{mis,i}=E(y_{mis,i}|\mathbf{y}_{-i})$
        \item Repeat 3. to 7. for $M$ times to obtain $M$ complete datasets.
    \end{enumerate}
\end{frame}

\begin{frame}{Imputation Model}
    \begin{itemize} 
        \item Generally, if one has no preliminary knowledge on the dataset, one should use as many variables as the covariate~\cite{rubin2018multiple}.
        \item In the survival setting, there has been always an argument on whether the $\log(t)$ should be used as the covariate or $H_0(t)$~\cite{white2009imputing}. 
        \item Therefore, we decided to include both scenarios in the simulation, to illustrate the best performed method. 
    \end{itemize}
\end{frame}

\begin{frame}{Simulation Study}
    \begin{table}
    \caption{The Simulation Results for 50\% missing on "PRS" covariate using Gamma frailty model, with 200 families with 3228 observations in total. $M=5$, * denotes a $p<0.05$. }
    \resizebox{\columnwidth}{!}{%
    \begin{tabular}{ccccccc}
    \hline
                            & Gamma Frailty & CCA         & \multicolumn{2}{c}{MI without Kinship}  & \multicolumn{2}{c}{MI with Kinship}  \\ 
                            & True value  &        &  $\log(T)$ &  $\log(H_0(T))$ &  $\log(T)$ & $\log(H_0(T))$ \\ \hline
    $\log(\alpha)$          & $-4.135$                  & $-3.869^*$ & $-4.269^*$                       & $-4.047^*$                       & $-4.320^*$                     & $-4.126^*$                    \\
    $\log(\lambda)$         & $1.099$                     & $1.125^*$  & $1.031^*$                         & $1.131^*$                        & $1.035^*$                      & $1.120$                       \\
    $\beta_{\text{gender}}$ & $1.000$                       & $1.112$    & $0.749$                           & $0.805$                          & $0.824$                        & $0.769$                       \\
    $\beta_{\text{mgene}}$  & $3.000$                     & $2.643$    & $\mathbf{2.760^*}$                         & $\mathbf{2.503^*}$                        & $\mathbf{2.648^*}$                      & $\mathbf{2.572^*}$                     \\
    $\beta_{\text{PRS}}$   & $3.000$                     & $2.703^*$  & $3.313^*$                         & $3.273^*$                        & $3.364^*$                      & $3.336^*$                     \\
    $\log(\kappa)$          & $0.693$                      & $0.651$    & $1.200$                           & $1.524$                          & $1.019$                        & $1.266$                       \\ \hline
    \end{tabular}%
    }
    \end{table}
\end{frame}

\begin{frame}{BRCA1 Data} 

    \begin{table}[]
    \caption{The Application Results for BRCA1 family data with 80\% missing rate on "PRS" variable using Gamma frailty model, with 498 families and 2650 individuals. $M=5$, * denotes a $p<0.05$. }
    \resizebox{\columnwidth}{!}{%
    \begin{tabular}{llllll}
    \hline
                           
                           & CCA       & \multicolumn{2}{c}{MI without Kinship}                                 & \multicolumn{2}{c}{MI with Kinship} \\
                           &           & $\log(T)$                          & $\log(H_0(T))$                    & $\log(T)$                       & $\log(H_0(T))$ \\ \hline
    $\log(\alpha)$         & $-7.344$  & $-4.463^*$                         & $-4.541^*$                        & $-4.601^*$                      & $-4.571^*$                     \\
    $\log(\lambda)$        & $2.424$   & $0.833^*$                          & $0.986^*$                         & $0.845^*$                       & $0.926^*$                      \\
    $\beta_{\text{mgene}}$ & $27.781$  & $2.064^*$                          & $2.211^*$                         & $2.149^*$                       & $2.269^*$                      \\
    $\beta_{\text{PRS}}$   & $11.434$  & $0.266$                            & $0.665$                           & $0.580$                         & $1.045$                        \\
    $\log(\kappa)$         & $-27.300$ & $1.083$                            & $0.799$                           & $1.046$                         & $1.001$                        \\ \hline
    \end{tabular}%
    }
    \end{table}
\end{frame}

\begin{frame}[t,allowframebreaks]{References}
    \bibliographystyle{unsrtnat}
    \bibliography{ref.bib}
\end{frame}

\begin{frame}{Advertisement}
    More Questions? Interested in being supervised by my fantastic supervisors? 

    My email: jbi23@uwo.ca
\end{frame}

\end{document}